\documentclass{article}

\usepackage[landscape,
paperwidth=3.26cm,
paperheight=19cm,
top=0.015cm,bottom=0cm,left=-0.27cm,right=-0.29cm,
width=14cm,
height=42cm]{geometry}
\usepackage{xeCJK}
../fonts/fontsetup.tex

\begin{document}
{\fontsize{50}{0}\texttt{zhang@linux.com}}
\newpage
\centering{\fontsize{100}{0}\selectfont 
\heiti{中国}\\
%\heiti{张在坤}\\
%\xinwei{张在坤}\\
%\mbox{\xingkai{故用兵之法,十则围之,五则攻之,倍则分之}}\\
%\mbox{\kaiti{故用兵之法,十则围之,五则攻之,倍则分之}}\\
%\mbox{\kaiti{凡治众如治寡,分数是也}}\\
%\mbox{\kaiti{《孙子兵法》}}\\
%\mbox{\kaiti{分而治之}}\\
%\mbox{\kaiti{师徒如父子;一日为师,终身为父。}}\\
%\mbox{\kaiti{师父}}\\
%\mbox{\kaiti{师兄}}\\
%\mbox{\kaiti{师伯}}\\
%\mbox{\kaiti{师叔}}\\
%\mbox{\kaiti{生日快乐,Toint 师伯!}}\\
%\mbox{\xinwei{谢谢!}}\\
%\mbox{\xingkai{谢谢!}}\\

%\kaiti{张在坤}\\
%\lishu{张在坤}\\
%\songti{张在坤}
}
\end{document}
